\documentclass{izpit}
\usepackage{fouriernc}

\begin{document}

\newcommand{\bnfis}{\mathrel{{:}{:}{=}}}
\newcommand{\bnfor}{\;\mid\;}
\newcommand{\fun}[2]{\lambda #1. #2}
\newcommand{\conditional}[3]{\mathtt{if}\;#1\;\mathtt{then}\;#2\;\mathtt{else}\;#3}
\newcommand{\recfun}[3]{\mathtt{rec}\;\mathtt{fun}\;#1\;#2. #3}
\newcommand{\tru}{\mathtt{true}}
\newcommand{\fls}{\mathtt{false}}
\newcommand{\tbool}{\mathtt{bool}}
\newcommand{\tand}{\mathbin{\mathtt{and}}}
\newcommand{\tandalso}{\mathbin{\mathtt{andalso}}}

\izpit
  {Teorija programskih jezikov: poskusni izpit}{6.\ januar 2020}{
  Čas pisanja je 150 minut. Možno je doseči 100 točk. Veliko uspeha!
}

%%%%%%%%%%%%%%%%%%%%%%%%%%%%%%%%%%%%%%%%%%%%%%%%%%%%%%%%%%%%%%%%%%%%%%%

\naloga[\tocke{25}]
Za vsakega od naslednjih izrazov ugotovite, ali ima tip in katerega. Nato ugotovite še, ali v operacijski semantiki malih korakov program divergira, se evalvira v vrednost, ali zatakne. Če se evalvira v vrednost, v katero?
\podnaloga $\conditional{(\conditional{\fls}{\fls}{\tru})}{\fls}{\tru}$ \prostor
\podnaloga $(\conditional{0=1}{\fls}{14}) * 3$ \prostor
\podnaloga $(\fun{g}{g \, \fls}) \, (\fun{x}{x + 1})$ \prostor
\podnaloga $\fun{b}{\conditional{b}{0}{1}}$ \prostor
\podnaloga $(\recfun{f}{b}{\conditional{f \, b}{42}{42}}) \, \fls$ \prostor

%%%%%%%%%%%%%%%%%%%%%%%%%%%%%%%%%%%%%%%%%%%%%%%%%%%%%%%%%%%%%%%%%%%
%%%%

\naloga[\tocke{25}]
Definirajmo izraz:
\[
  e = \fun{f}{\fun{g}{\fun{x}{f(g(f x))}}}
\]
\podnaloga Poiščite tip $A$ in sistem enačb $\mathcal{E}$, tako da velja $\emptyset \vdash e : A | \mathcal{E}$ ter zapišite ustrezno drevo izpeljave.
\prostor[3]
\podnaloga Poiščite najbolj splošno rešitev sistema $\mathcal{E}$, torej substitucijo $\sigma$, za katero velja $\mathcal{E} \searrow \sigma$.\prostor


%%%%%%%%%%%%%%%%%%%%%%%%%%%%%%%%%%%%%%%%%%%%%%%%%%%%%%%%%%%%%%%%%%%%%%%

\naloga[\tocke{25}]
V $\lambda$-račun dodamo operaciji:
\[
  e \bnfis \cdots \bnfor
  e_1 \tand e_2 \bnfor
  e_1 \tandalso e_2
\]
Obe operaciji naj bi izračunali logično konjunkcijo Boolovih izrazov $e_1$ in $e_2$, razlika je le v tem, da $\mathtt{and}$ evaluira~$e_2$ samo po potrebi, če iz~$e_1$ ni razviden rezultat, medtem ko $\mathtt{andalso}$ vedno evaluira oba~$e_1$ in~$e_2$.

\podnaloga Zapišite pravila za operacijsko semantiko in določanje tipov za $\mathtt{and}$ in $\mathtt{andalso}$.\prostor[2]
\podnaloga Podajte primer izrazov~$e_1$ in~$e_2$ tipa $\tbool$,
  iz katerih je opazna razlika med $e_1 \tand e_2$ in $e_1 \tandalso e_2$.\prostor
\podnaloga Dokažite, da za razširjeni jezik še vedno velja izrek o varnosti.\prostor[2]

%%%%%%%%%%%%%%%%%%%%%%%%%%%%%%%%%%%%%%%%%%%%%%%%%%%%%%%%%%%%%%%%%%%%%%%

\naloga[\tocke{25}]
Običajni programski jeziki omogočajo medsebojne rekurzivne definicije, na primer v OCamlu lahko definiramo:
\begin{verbatim}
  let rec sodo n =
    if n = 0 then true else (not (liho (n - 1)))
  and liho n =
    if n = 0 then false else (not (sodo (n - 1)))
\end{verbatim}

\podnaloga
S pomočjo običajnega izreka o negibnih točkah zveznih preslikav na domenah dokažite, da za poljubni domeni $D$ in $E$ ter zvezni funkciji $f \colon D \times E \to D$ in $g \colon D \times E \to E$ obstajata fiksni točki $x \in D$ in $y \in E$, da velja:
\begin{align*}
  x &= f(x, y) &
  y &= g(x, y)
\end{align*}
\prostor

\podnaloga
Naj bosta $s, \ell \colon \mathbb{N}_\bot \to \mathbb{B}_\bot$ preslikavi, ki zadoščata rekurzivnim enačbam:
\begin{align*}
  s(\bot) &= \bot &
  s(0) &= \mathrm{tt} &
  s(n) &= n (\ell(n - 1)) \\
  \ell(\bot) &= \bot &
  \ell(0) &= \mathrm{ff} &
  \ell(n) &= n (s(n - 1))
\end{align*}
kjer je preslikava $n \colon \mathbb{B}_\bot \to \mathbb{B}_\bot$ podana z
\begin{align*}
  n(\bot) &= \bot &
  n(\mathrm{tt}) &= \mathrm{ff} &
  n(\mathrm{ff}) &= \mathrm{tt}
\end{align*}
Poiščite domeni $D$ in $E$ ter zvezni funkciji $\Phi \colon D \times E \to D$ in $\Psi \colon D \times E \to E$, tako da velja
\begin{align*}
  s &= \Phi(s, \ell) &
  \ell &= \Psi(s, \ell)
\end{align*}
Tega, da sta $D$ in $E$ domeni ter $\Phi$ in $\Psi$ zvezni, vam ni treba dokazovati.
\prostor

\end{document}
